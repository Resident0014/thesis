%% Не менять - Do not modify
%%\input{my_folder/summary_settings} 
\chapter*[Count-me]{Реферат} % * - не нумеруем
\thispagestyle{empty}% удаляем параметры страницы
%\setcounter{sumPageFirst}{\value{page}}
%sumPageFirst \arabic{sumPageFirst}
%
%
%% Возможность проверить другие значения счетчиков - debugging
%\ref*{TotPages}~с.,
%\formbytotal{mytotalfigures}{рисун}{ок}{ка}{ков},
%\formbytotal{mytotaltables}{таблиц}{у}{ы}{},
%There are \TotalValue{mytotalfigures} figures in this document
%There are \TotalValue{mytotalfiguresInApp} figuresINAPP in this document
%There are \TotalValue{mytotaltables} tables in this document
%There are \TotalValue{mytotaltablesInApp} figuresINAPP in this document
%There are \TotalValue{myappendices} appendix chapters in this document
%\total{citenum}~библ. наименований.



%% Для того, чтобы значения счетчиков корректно отобразились, необходимо скомпилировать файл 2-3 раза
На \total{mypages}~c.,
\total{myfigures},
\total{myappendices}\\


{\MakeUppercase{Ключевые слова: \keywordsRu}.}\\ % Ключевые слова из renames.tex

Тема выпускной квалификационной работы: <<\thesisTitle>>.


\abstractRu % Аннотация из renames.tex
\newpage



\printTheAbstract\\ % не удалять


\total{mypages}~pages, 
\total{myfigures}~figures, 
%\total{mytables}~tables,
\total{myappendices}~appendices\\%.

%\noindent
{\MakeUppercase{Keywords: \keywordsEn}.}\\ % Ключевые слова из renames.tex 

The subject of the graduate qualification work is <<\thesisTitleEn>>.

	
\abstractEn % Аннотация из renames.tex

	


%% Не менять - Do not modify
\thispagestyle{empty}
%\setcounter{sumPageLast}{\value{page}} %сохранили номер последней страницы Задания
%\setcounter{sumPages}{\value{sumPageLast}-\value{sumPageFirst}}
%sumPageLast \arabic{sumPageLast}
%
%sumPages \arabic{sumPages}
%\restoregeometry % восстанавливаем настройки страницы
%\input{my_folder/summary_settings_restore}	% настройки - конец
\newpage