\chapter{Постановка задачи и разработка требований к системе} \label{ch3}
В данной главе производится постановка задачи и разработка требований к сервису интерактивной визуализации моделей программ.
\section{Постановка задачи} \label{ch3:sec1}
Исходя из анализа существующих решений можно сделать вывод, что не существует интерактивного визуализатора, отображающего все заявленные модели программ, поэтому необходимо разработать систему для интерактивной визуализации моделей кода на языке Java, а именно:
\begin{itemize}
\item AST
\item CFG
\item DDG
\item PDG
\item ASG
\item SSA
\end{itemize}

Для выполнения этой задачи требуется:
\begin{itemize}
\item Разработать требования к системе
\item Спроектировать архитектуру сервиса
\item Проанализировать и выбрать средства решения
\item Разработать сервис в соответствии с установленными требованиями
\item Протестировать и отладить разработанный сервис
\end{itemize}
\section{Разработка требований к системе} \label{ch3:sec2}

Необходимо разработать сервис, позволяющий получать код метода на языке Java от пользователя и строить по этому коду визуализацию выбранной модели.

Общие требования:
\begin{itemize}
\item Сервис должен быть интерактивным
\end{itemize}

Любое изменение кода должно приводить к изменению модели без дополнительных действий со стороны пользователя. Если введённый код некорректен, сервис должен сообщить пользователю об этом.
\begin{itemize}
\item Сервис должен работать на компьютерах, смартфонах и планшетах
\end{itemize}

Должна быть возможность воспользоваться сервисом с любых устройств, на которых есть браузер.
\begin{itemize}
\item Сервис должен позволять строить все обозначенные модели через единый интерфейс
\end{itemize}

Это AST, CFG, DDG, PDG, ASG и SSA. У пользователя должна быть возможность переключаться между моделями.
\begin{itemize}
\item Удобная навигация по большим моделям
\end{itemize}

Должна быть возможность изменения масштаба изображения модели, а также возможность его перемещения внутри области просмотра.
\begin{itemize}
\item Элементы графов должны размещаться детерминировано
\end{itemize}

Для одного и того же кода необходимо всегда получать одинаковую визуализацию всех моделей.

Уточняющие требования по AST:
\begin{itemize}
\item В дереве должны присутствовать только семантически значимые узлы
\item Должна быть возможность сворачивать и разворачивать поддеревья
\end{itemize}

Уточняющие требования по CFG:
\begin{itemize}
\item Необходимо использовать стандартные формы для обозначения узлов
\end{itemize}

Эллипс для обозначения начала или конца, прямоугольник для обычных операторов, ромб для узлов принятия решений.
\begin{itemize}
\item Метки на рёбрах из узлов принятия решений
\end{itemize}

Рёбра, выходящие из узлов принятия решений, должны иметь обозначения, дающие понять пользователю, по какому ребру переходить в случае, если условие истинно, а по какому в случае, если ложно.
\begin{itemize}
\item Граф не должен быть привязан к языку программирования
\end{itemize}

Нужно убрать типы данных из названий узлов, а также объявления переменных без инициализации. Конструкции if, for и while должны быть отображены как условия с необходимыми стрелками. Инструкции break и continue не должны появляться как самостоятельные узлы, а должны только влиять на расположение стрелок.
\section{Выводы по главе} \label{ch3:sec3}
В этой главе была поставлена задача, которую необходимо решить в рамках выпускной квалификационной работы и определены требования для реализации сервиса интерактивной визуализации моделей программ.
\newpage




% не рекомендуется использовать отдельную section <<введение>> после лета 2020 года
%\section{Введение} \label{ch3:intro}

%Хорошим стилем является наличие введения к главе. Во введении может быть описана цель написания главы, а также приведена краткая структура главы. 
%	
%\section{Название параграфа} \label{ch3:sec1}
%
%\section{Название параграфа} \label{ch3:sec2}
%
%%\FloatBarrier % заставить рисунки и другие подвижные (float) элементы остановиться
%
%
%\section{Выводы} \label{ch3:conclusion}
%
%Текст выводов по главе \thechapter.
%
%
%%% Вспомогательные команды - Additional commands
%%
%%\newpage % принудительное начало с новой страницы, использовать только в конце раздела
%%\clearpage % осуществляется пакетом <<placeins>> в пределах секций
%%\newpage\leavevmode\thispagestyle{empty}\newpage % 100 % начало новой страницы