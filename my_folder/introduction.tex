\chapter*{Введение} % * не проставляет номер
\addcontentsline{toc}{chapter}{Введение} % вносим в содержание

На ранних этапах обучения программированию студенты зачастую сталкиваются с непониманием программного кода. Оно, отчасти, происходит из-за отсутствия визуализации. Согласно исследованию, 65\% информации человек воспринимает зрением, а остальные 35\% всеми другими органами чувств. Визуализатор значительно упростит как изложение материала преподавателям, так и восприятие получаемой информации студентами. Визуализатор – вид программного обеспечения, предназначенный для преобразования разной информации в зрительные образы. Может быть или отдельным приложением, или плагином, или частью иного приложения. На данный момент существует множество таких утилит, например, визуализатор поведения многопоточных Java-программ, который демонстрирует выполнение написанного кода в виде схемы, в то время как без визуализации понять некоторые программы, в которых используется многопоточность, затруднительно. Это происходит из-за того, что исходный код, представленный в виде текста, достаточно плохо отображает ход выполнения программы. 
Такие области программирования, как графы и деревья, являются достаточно сложными в представлении и понимании, поэтому визуализатор, разрабатываемый в ходе данной выпускной работы, будет являться наглядным инструментом для демонстрации моделей программ, включающих в себя одну функцию или метод, написанный на языке программирования Java. Проблема визуализации графов весьма старая и для ее решения было разработано много средств, однако практически все найденные решения направлены на визуализацию одного типа графа, написанного на определенном языке программирования. Цель данной работы - создать интерактивный визуализатор моделей программ, то есть наглядного инструмента для демонстрации таких моделей программ, как абстрактного синтаксического дерева (AST), графа потока управления (CFG), графа зависимости по данным (DDG), графа зависимости программы (PDG) и абстрактного семантического графа (ASG) и представления в виде однократного статического присваивания (SSA). Визуализатор поддерживает код, написанный на Java. 
В данной выпускной квалификационной работе будут рассмотрены существующие программные решения для визуализации, проведен их обзор и анализ. Результатом работы является интерактивное клиент-серверное приложение, выполняющее все поставленные задачи. В конце работы проводится тестирование и анализ результатов.  


%% Вспомогательные команды - Additional commands
%\newpage % принудительное начало с новой страницы, использовать только в конце раздела
%\clearpage % осуществляется пакетом <<placeins>> в пределах секций
%\newpage\leavevmode\thispagestyle{empty}\newpage % 100 % начало новой строки