\chapter*{Заключение} \label{ch-conclusion}
\addcontentsline{toc}{chapter}{Заключение}	% в оглавление 
В рамках выпускной квалификационной работы был разработан сервис интерактивной визуализации моделей программ, как средство для улучшения понимания устройства абстрактного синтаксического дерева, графа потока управления, графа зависимостей по данным, графа зависимости программы, абстрактного семантического графа и представления на основе однократного статического присваивания. 
Сперва были рассмотрены аналоги, их плюсы и минусы. Исходя из обзора был сделан вывод, что на данным момент не существует универсального и хорошего продукта, у всех есть какие-либо недостатки, так же не существует ни одного сервиса, который мог бы построить несколько заявленных моделей. После разбора были получены требования к системе, выбрана библиотека для рисования, парсер кода и фреймворк для сервера и клиента и была описана разработка. Итоговая версия визуализатора имеет клиент-серверную архитектуру, сервер написан на языке программирования Java, использует парсер JavaParser и фреймворк Spark. Клиент же написан на языке программирования JavaScript с использованием фреймворка Vue.js, библиотекой axios для сетевых запросов, фреймворка bulma для оформления интерфейса и использует библиотеку d3-graphviz для визуализации моделей.
Сервис был протестирован за счет анализа ряда методов, написанных на Java. Ожидаемые деревья и графы совпали с теми, которые рисует визуализатор и сейчас им можно пользоваться в образовательных целях. В будущем планируется добавление других языков программирования и расширение функционала.
\newpage